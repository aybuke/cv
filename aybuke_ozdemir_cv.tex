\documentclass[margin,line]{res}

\topmargin=-0.8in
\oddsidemargin -.5in
\evensidemargin -.5in
\textwidth=6.0in
\itemsep=0in
\parsep=0in
\textheight=12in

\newenvironment{list1}{
  \begin{list}{\ding{113}}{%
      \setlength{\itemsep}{0in}
      \setlength{\parsep}{0in} \setlength{\parskip}{0in}
      \setlength{\topsep}{0in} \setlength{\partopsep}{0in}
      \setlength{\leftmargin}{0.1in}}}{\end{list}}
\newenvironment{list2}{
  \begin{list}{$\bullet$}{%
      \setlength{\itemsep}{0in}
      \setlength{\parsep}{0in} \setlength{\parskip}{0in}
      \setlength{\topsep}{0in} \setlength{\partopsep}{0in}
      \setlength{\leftmargin}{0.1in}}}{\end{list}}


\begin{document}
\small
\name{\Large Ayb\"{u}ke \"{O}zdemir Kaya\vspace*{.1in}}

\begin{resume}
\section{\sc \.{I}let\.{I}\c{s}\.{I}m B\.{I}lg\.{I}ler\.{I}}
\begin{tabular}{@{}p{2in}p{3.8in}}
aybuke.147@gmail.com & {\hfill{\it}  +90 (554) 343-2144} \\
\end{tabular}
\begin{tabular}{@{}p{2in}p{3.8in}}
www.aybukeozdemir.com & {\hfill{\it}  github.com/aybuke} \\
\end{tabular}

\vspace*{-.3cm}
\section{\sc E\u{g}\.{I}t\.{I}m}

{\bf \c{C}anakkale 18 Mart \"{U}niversitesi}, \c{C}anakkale, T\"{u}rkiye  \hfill {\bf 2011 - 2016} \\

\vspace*{-.7cm}
{\em Lisans, Bilgisayar M\"{u}hendisli\u{g}i} \hfill {\ } \\
\vspace*{-.09in}
\vspace*{-.3cm}


\vspace*{-.3cm}
\section{\sc Deney\.{I}mler}

{\bf VNGRS, Devops Engineer}, \.{I}stanbul, T\"{u}rkiye \hfill {\bf May{\i}s 2018 - \c{S}imdi}\\

\vspace{-.7cm}
{\em Devops Tak{\i}m{\i} }
\vspace*{+.05in}
\begin{list2}
\item AWS, Azure ve Google Cloud \"{u}zerinde \c{c}al{\i}\c{s}ma f{\i}rsat{\i}m oldu. Cloud servisleri d{\i}\c{s}{\i}nda kubernetes, terraform, ansible gibi teknolojileri kulland{\i}m.
\end{list2}

{\bf EBrandValue, Devops Engineer}, \.{I}stanbul, T\"{u}rkiye \hfill {\bf Kas{\i}m 2016 - May{\i}s 2018}\\

\vspace{-.7cm}
{\em Devops Tak{\i}m{\i} }
\vspace*{+.05in}
\begin{list2}
\item \c{S}imdiye kadar ansible, terraform ve Dockerfile scriptleri yazd{\i}m. Marathon/Mesos ile container y\"{o}netimi deneyimim oldu. Ayn{\i} zamanda Python/Django(Backend) geli\c{s}tirmelerinde ve oncall rotasyonunda aktif rol ald{\i}m.

\end{list2}

{\bf Metglobal, Stajyer}, \.{I}stanbul, T\"{u}rkiye \hfill {\bf Temmuz 2014}\\

\vspace{-.7cm}
{\em Sistem Y\"{o}netimi B\"{o}l\"{u}m\"{u} }
\vspace*{+.05in}
\begin{list2}
\item Puppet Policy-Base Autosigning i\c{c}in bir sorgulama beti\u{g}i yazd{\i}m. (https://github.com/aybuke/policy-based-for-PUPPET).
\end{list2}

{\bf Gamegos, Stajyer}, \.{I}stanbul, T\"{u}rkiye \hfill {\bf A\u{g}ustos 2013}\\

\vspace{-.7cm}
{\em Sistem Y\"{o}netimi B\"{o}l\"{u}m\"{u} }
\vspace*{+.05in}
\begin{list2}
\item Haproxy yap{\i}land{\i}rma dosyas{\i} i\c{c}in Python k\"{u}t\"{u}phanesi ve bu k\"{u}t\"{u}phane i\c{c}in de bir Rest api yazd{\i}m. (Projeler: https://github.com/aybuke/hapra ve https://github.com/aybuke/python-haproxy-tools).
\end{list2}

\section{\sc Projeler ve Katk{\i}lar}
\begin{list2}
\item {\bf LibreOffice: Ofis paketi i\c{c}in yapt{\i}\u{g}{\i}m katk{\i}lar linktedir.}

https://gerrit.libreoffice.org/\#/q/aybuke+status:merged
\item {\bf Outreachy 2014-2016: Linux Kernel i\c{c}in yapt{\i}\u{g}{\i}m katk{\i}lar linktedir.}

https://patchwork.kernel.org/project/opw-kernel/list/?submitter=90241\&archive=both
\item {\bf Mozilla Balrog Projesi katk{\i}c{\i}s{\i}y{\i}m:} Balrog, Mozilla'n{\i}n uygulama g\"{u}ncelleme servisidir.(AUS)

https://github.com/mozilla/balrog-ui/
\item {\bf LibreOffice Crash Projesi katk{\i}c{\i}s{\i}y{\i}m:}  LibreOffice sunucular{\i}ndaki \c{c}\"{o}kme raporlamalar{\i} i\c{c}in kullan{\i}l{\i}yor.

https://github.com/mmohrhard/crash/
\item {\bf Pebble Ak{\i}ll{\i} Saat Uygulamam Magic Button:}

https://github.com/pebble-tr/Magic-Button
\item {\bf Pebble Watchface Textwatch-tr geli\c{s}tiricilerindenim:}

https://github.com/pebble-tr/PebbleTextWatch-tr
\item {\bf LibreOffice ve Mozilla \c{c}eviri tak{\i}m{\i}nday{\i}m:}

https://support.mozilla.org/tr/kb/locales/tr

https://wiki.documentfoundation.org/Language/\c{C}evirmenler/tr
\end{list2}
\vspace{-.3cm}

\section{\sc Sunumlar ve Sert\.{I}f\.{I}kalar}
\begin{list2}
\item LPIC1 ve OpenSuse Sertifikas{\i} (id: LPI000278784)
\item Amat\"{o}r Telsizci Sertifikas{\i} (TA3IOQ)
\item Pythonsaati 46 Virtualization at the Application Layer (2018)
\item Pythonsaati 36 Infrastructure as Code (2018)
\item \"{O}zg\"{u}r Yaz{\i}l{\i}m ve Linux G\"{u}nleri (2016) 	LibreOffice Geli\c{s}tirme ve Yayg{\i}nla\c{s}t{\i}rma Toplant{\i}s{\i}
\item Akademik Bili\c{s}im Konferans{\i} - Ayd{\i}n \"{U}niversitesi (2016) LibreOffice \c{C}al{\i}\c{s}ma At\"{o}lyesi (4 g\"{u}n)
\item Akademik Bili\c{s}im Konferans{\i} - Ayd{\i}n \"{U}niversitesi (2016) “Kad{\i}nlar \.{I}\c{c}in \"{O}zg\"{u}r Yaz{\i}l{\i}m F{\i}rsatlar{\i}
\item Bal{\i}kesir \"{U}niversitesi (2016) \"{O}zg\"{u}r Yaz{\i}l{\i}m ve \"{O}zg\"{u}r Toplum Semineri
\item Devfest \.{I}stanbul (2015) LibreOffice'e Katk{\i} S\"{u}reci ve Deneyimlerimiz
\item INET-TR (2015) Kad{\i}n Bili\c{s}im Paneli 
\item DevFest Student Eski\c{s}ehir (2014) “Linux \c{C}ekirde\u{g}ine Katk{\i} S\"{u}reci ve Deneyimlerimiz
\item Mu\u{g}la 2. \"{O}zg\"{u}r Yaz{\i}l{\i}m \c{C}al{\i}\c{s}tay{\i} (2014) “Linux \c{C}ekirde\u{g}ine Katk{\i} S\"{u}reci ve Deneyimlerimiz
\item \"{O}zg\"{u}r Web Teknoloji G\"{u}nleri (2014) “Puppet Sertifika Y\"{o}netimi
\item Akademik Bili\c{s}im Konferans{\i} - Mersin \"{U}niversitesi (2014) “\"{U}niversite \"{O}\u{g}rencileri \.{I}\c{c}in \"{O}zg\"{u}r Yaz{\i}l{\i}m F{\i}rsatlar{\i}
\item \"{O}zg\"{u}r Web Teknoloji G\"{u}nleri (2013) “Haproxy ve Hapra Nedir?
\end{list2}
\vspace{-.3cm}

\section{\sc Yetenekler}
\begin{list2}
\item Diller: T\"{u}rk\c{c}e, \.{I}ngilizce (B2)
\item Programlama: Python, C/C++, Go {\em biraz Perl ve Ruby}.
\item Ara\c{c}lar: Vim, Git, Gerrit, Mercurial, SSH, Screen, Terraform, Kubernetes, Helm, Ansible, Jenkins
\item Frameworks: Django, Tornado, Flask, Chalice
\end{list2}

\vspace{-.3cm}
\section{\sc Hob\.{I}ler}
\begin{list2}
\item \"{O}zg\"{u}r yaz{\i}l{\i}m i\c{c}in katk{\i} vermek.
\item Roman-dergi okumak, dizi-film izlemek.
\item Seyahat etmek.
\item Davul \c{c}almak, \c{c}izim yapmak.
\end{list2}

\end{resume}
\end{document}
